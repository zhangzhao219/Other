\pagestyle{fancy}
\section{Article of results and discussions about fungi}

{\fontsize{14pt}{25pt}\selectfont 

We are the participating team of 2021 MCM Contest. In this contest, we chose the issue of "Fungi". In the process of solving this problem, we conducted in-depth exploration and analysis, established relevant models and finally successfully solved these problems. At the same time, we have gained a deeper understanding of the role of fungi in ecosystems and their interaction with the environment.

Decomposers are an essential part of the carbon cycle in the earth's circle, and fungi are an important part of the decomposers. The important role of fungi is to decompose dead branches and wood fibers. Therefore, it is of great significance to study the effects of fungi and ensure the decomposition rate of wood fibers to maintain the normal progress of the carbon cycle.

While the previous paragraph described only one side of fungi that we all know, the conclusions from our model analysis can help us to further understand and understand different sides of fungi and their interaction with the environment.

Our model analysis and conclusions are summarized as follows: temperature, humidity, weather and climate, fungal species and the species combination of different fungal species will affect the overall lignin fiber decomposition rate; Environmental changes have a certain effect on the decomposition rate of fungi, which can be resisted or eliminated by the fungus-environment system when it is not very serious. To some extent, the diversity of the fungal community determines the resistance and self-recovery ability of the system to environmental changes. The higher the species diversity, the stronger the resistance ability of the system is, and the easier it is to recover when damaged.

First, the above analysis suggests that fungal diversity plays a crucial role in maintaining fungal decomposition. When local or short-term changes occur in the environment, the fungus-environment system can maintain the normal decomposition through the self-regulation within the system without external intervention, which is a sustainable environmental system.

Second, we should prevent serious natural disasters such as fires and floods and major physical and chemical damage such as chemical leaks. Such accidents can cause huge environmental changes that exceed the system's ability to self-regulate, destroy fungal diversity and seriously affect the decomposition of wood fibers and the carbon cycle within the region, thereby threatening the normal activities of other biological environments. If the post-disaster reconstruction of the ecosystem is to be carried out, attention should be paid to the combination of fungus species and the scale of fungus release when releasing fungi, because the initial conditions in the model determine the final stability of the model.

At the same time, we should pay attention to prevent the invasion of foreign fungal species. An increase in fungal species does not necessarily increase species diversity. Species diversity can be increased only when the interaction between fungi satisfies the requirement of system stability.

Finally, through this study on fungi, we realize the importance of making more people aware of the conservation of species diversity. If the decomposers are destroyed, the ecosystem will be destroyed. As a member of the ecosystem, we humans are not immune. Our normal production and life will also be seriously affected. Therefore, the protection of biodiversity is closely related to each of us. We must pay attention to it and take practical actions to protect our homes.

}
