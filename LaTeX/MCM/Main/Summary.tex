\begin{abstract}
    
In order to analyze the decomposition of woody fibers in fungal community, this paper deduces the competitive formula applicable to multi-fungal community from the relevant theory, and regards this problem as a problem of multi-interaction and coexistence between populations based on competition. We establish and optimize the relationship between populations, focus on analyzing the stability of the model, and finally simulate the relationship between fungal community, diversity and the influence of environmental condition fluctuation on the overall rate decomposition of fungi reasonably. The stable conditions of the system are determined theoretically.

For problem 1 and 2, we use the linear fitting method, and show the effect of different arguments on the rate of fungal decomposition as a function. For the two most important variables, Hyphal extension and moisture trade-off. By researching the relevant images, we carry out linear and nonlinear fitting method, obtain the relationship and merge them .

For problem 3, we refine our model based on the Lotuka-Voltera model, and the relationship between different fungi is expressed through the different equations. To solve the singerum characteristics and types and analyze the stability and stable conditions of the model, we finally give an analysis of the two fungal community as examples and list the competitive results under different circumstances. It is concluded that when the two sides only have a competitive relationship, the two will eventually achieve relative stability in quantity. However, if there is a suppression relationship, there will be a situation in which one of the winning parties is eliminated.

For problem 4, which based on the model of problem 3, the effects of dynamic fluctuations are added to simulate short-term changes in different environments such as drought, semi-arid, temperate, trees and tropical rainforests. We analyze the stability of the power system in the presence of dynamic interference, and divide the results of each fungus into four cases: “extinction”, “average persistent”, “weakly average persistent”, and “strongly average persistent” according to relevant definitions. Furthermore we analyze the each situation needs to meet the conditions.

For problem 5, in order to simulate fungal diversity, which includes competition, inhibition, promotion and the existence of three interrelations at the same time, we optimize the model further taking the three fungal coexistence as an example. We use the matrix to analyze the diversity system, so the conditions of the positive equilibrium point of global progressive stability are clearly expressed in matrix form, which describes the important influence of fungal diversity on ecosystem stability and system self-regulation ability. Finally, the overall decomposition rate of wood fibers of the diversity fungus system is solved by combining the models of problem 1 and 2.

\begin{keywords}
    The decomposition rate of fungi, Stability analysis, Ecosystem diversity, Competition-inhibition-promotion mixed relationship model, Dynamic system, Random disturbance
\end{keywords}
\end{abstract}
\maketitle
    